%!TEX root = ../SciVis.tex

This document is part of the Scientific Visualization course of the Computing Science department of the University of Groningen. The purpose of this document is to describe the steps that have been followed in order to realise a Real-time simulation of fluid flows.

Implementing a visualisation software is a rather complex process. Many things need to be taken into consideration such as obtaining a clear understanding of the simulation or the real-time phenomenon to be visualised, finding the most appropriate visualisation techniques in order to be able to extract all the desired information, and finally continue with the implementation step. This is roughly the approach that we followed in order to realise this project.

We needed to obtain a very good understanding of what we wanted to visualise, which are the properties of a fluid flow, and what we could expect, for example by applying a force in a particular point. The visualisation techniques were given to us as concrete steps by the professor, so we were able to continue with the implementation and along with it, to gain a deeper knowledge about the visualisation techniques and what information we could extract by using each of them.\\

The document is structured as follows. Section~\ref{sec:implementation} describes in detail all the steps that we followed during the implementation of this software. Section~\ref{sec:skeletonCompilation} briefly explains how we set up the source code so we were able to continue with implementation. Section~\ref{sec:colorMapping} describes various color mapping techniques that we used to extract information from the fluid flow phenomenon. Color mapping is one of the most important parts of a visualisation software and this step of the project was one of the most difficult and time-consuming to implement.

Section~\ref{sec:glyphs} describes the various glyph icons that we have chosen to implement and to use as part of the vector visualisation part. Section~\ref{sec:divergence} describes in detail the divergence operator that we apply to the vector fields of the fluid flow. Then in Section|~\ref{sec:isolines}, the implementation of the technique of isolines is described. The sixth step was the implementation of another visualisation technique, called height plots and it is described in Section~\ref{sec:heightplots}. The next and final step was to implement stream tubes. This visualisation technique is presented in Section~\ref{sec:streamtubes} and  it was the most difficult and time-consuming step of the implementation.